\documentclass[11pt,a4j]{jarticle}
% \usepackage[dvipdfmx]{graphicx}
\usepackage{multicol}
\usepackage{amssymb}
\usepackage[dvipdfmx]{graphicx}
\usepackage{here}
\usepackage{listings,jvlisting} %日本語のコメントアウトをする場合jvlisting(もしくはjlisting)が必要
%ここからソースコードの表示に関する設定
\lstset{
  basicstyle={\ttfamily},
  identifierstyle={\small},
  commentstyle={\smallitshape},
  keywordstyle={\small\bfseries},
  ndkeywordstyle={\small},
  stringstyle={\small\ttfamily},
  frame={tb},
  breaklines=true,
  columns=[l]{fullflexible},
  numbers=left,
  xrightmargin=0zw,
  xleftmargin=3zw,
  numberstyle={\scriptsize},
  stepnumber=1,
  numbersep=1zw,
  lineskip=-0.5ex
}
%ここまでソースコードの表示に関する設定


\title{双安定マルチバイブレータ}
\date{}

\begin{document}
    \section{はじめに}
    トランジスタを用いた双安定マルチバイブレータを設計・製作し,外部からトリガーパルスを周期的に入力したときの各部の波形を測定して,回路の動作を明確に理解することを目的とする.
    \section{課題1(組合せ回路の設計)}
        セルフバイアス型双安定マルチバイブレータの回路を図1に示す.
        \subsection{半加算器}
        \subsection{全加算器}
        \subsection{4ビット並列加算器}
            \subsubsection{課題内容}
            \subsubsection{設計結果}
            \subsubsection{実機検証}
            \subsubsection{シミュレーション}
                本実験においては,テストベンチでカバーしなければならない各PINへの入力・出力の組み合わせが2,560ケースに上った.そこで,すべてのケースをカバーするとともに,理論値との比較がコンピュータで行えるよう,テストベンチ並びに理論値データの自動生成を行うPythonスクリプトを作成した.また,シミュレーション結果の波形を目視で検証するコストを削減するため,テストベンチコードが「入力が入ってから3ns経過地点での各出力値」を理論値データと同じフォーマットでModelSimのTranscriptに出力するように設計した.
                
                付録に,テストベンチを生成したスクリプト,生成されたテストベンチコード,生成された理論値データ,シミュレーション結果の波形を記載する.

                理論値を算出するプログラム部分のみを抜粋して以下に掲載する.

                \begin{lstlisting}[caption=hoge,label=fuga]

#############################
# statics
...
ARRAYSTR = ["Theoretrical Value", "A3", "A2", "A1", "A0", "B3", "B2", "B1", "B0", "C3", "S3", "S2", "S1", "S0"]
...
#############################
# Calucurating Theoretrical Value
for pattern in patternList:
    A = int("".join(map(str, pattern[0:4])),2)
    B = int("".join(map(str, pattern[4:8])),2)
    if LOG_LEVEL >= 2:
        print("A = " + str(A) + " / A = " + str(pattern[0:4]) + " / B = " + str(B) + " / B = " + str(pattern[4:8]), end="")
    pattern.insert(0, A + B)
    if LOG_LEVEL >= 2:
        print(" / Cal = " + str(A+B))
...
#############################
# Generating Test Code
resultFile = open(RESULT_PATH, mode="w")
for i in patternList:
    if LOG_LEVEL >= 1:
        for p in range(1,9):
            print(ARRAYSTR[p] + ":[\'" + str(i[p]) + "\'] ", file=resultFile, end="")
        caledResVal = str(bin(i[0])).replace("b","").zfill(5)
        if len(caledResVal) > 5:
            caledResVal = caledResVal[1:6]
        print(caledResVal)
        for q in range (9,14):
            print(ARRAYSTR[q] + ":[\'" + str(caledResVal[q-9:q-8]) + "\'] ", file=resultFile, end="")

        print("\n", file=resultFile, end="")
        print("A = " + str(int("".join(map(str, i[1:5])),2)) + " / A = " + str(i[1:5]) + " / B = " + str(int("".join(map(str, i[5:9])),2)) + " / B = " + str(i[5:9]) + " / cal = " + str(i[0]))

                \end{lstlisting}

                結果,理論値データである"fourbitcounter\_tb.result"と,シミュレーション結果である"simulatedResult"を入手した.これをLinuxにデフォルトで実装されているdiffコマンドにて差分確認したところ,差分は見つからなかった.これにより,正常に設計されていることが確認できた.

                \begin{lstlisting}[caption=diffコマンドの実行結果,label=fuga]
[root@XXX result]# diff fourbitcounter_tb.result simulatedResult 
[root@XXX result]# 
                \end{lstlisting}








\end{document}